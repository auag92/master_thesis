\documentclass[a4paper,10pt]{article}
\usepackage[utf8]{inputenc}
% \usepackage{amsmath}
\usepackage{amsfonts, amsmath, amsthm, amssymb} % For math fonts, symbols and environments
%opening
\title{Examination: Modeling and simulation methods in materials science (MT218) (Max: 38 marks)}
\author{}

\begin{document}

\maketitle

\section{Numerical methods}
\begin{itemize}
 \item Using the taylor series expansion work out a laplacian operator $\dfrac{\partial ^{2}}{\partial x^{2}}$ 
 that is second order accurate in space. (2marks)
 \item For the classical diffusion equation, $\dfrac{\partial c}{\partial t} = D\dfrac{\partial ^{2}}{\partial x^{2}}$,
 write down how one would implement the periodic and no-flux boundary conditions in the explicit
 finite difference setting, using this operator. (2mark)
 \item In a small sketch highlight what would be the "actual boundary" where the boundary condition is applicable. (1mark)
\end{itemize}

\begin{itemize}
 \item For the Allen-Cahn equation $\dfrac{\partial \eta}{\partial t} 
 = M\left(2\kappa \dfrac{\partial^{2}\eta}{\partial x^{2}} - \dfrac{\partial f}{\partial \eta}\right)$, write down
 the set of equations that you would derive for an implicit finite difference discretization for a 1D setting
 consisting of 4 real points, using the three-point stencil for the laplacian (the same that you constructed 
 in the first question). (3marks)
 \item Work out the solution to this matrix, using the Thomas-Algorithm, with Dirichlet boundary conditions 
 for one time-iteration update.(2marks)
\end{itemize}

\section{Cahn-Hilliard model}
\begin{itemize}
 \item For an equilibrium interface profile of a binary alloy, for the Cahn-Hilliard model discussed in class
 draw the following, a) the composition profile(mention the solution), b) ($\dfrac{\partial f}{\partial c}$),
 c) $f-\left(\dfrac{\partial f}{\partial c} - 2\kappa \nabla^{2}c\right)c$, d) 
 $\left(\dfrac{\partial f}{\partial c} - 2\kappa \nabla^{2}c\right)$.(4marks)
 \item Write the equi-partition relation which gives the partial differential equation for the equilibrium interface. (1mark)
 \item For the free-energy density of the form $f(c)=c^{2}\left(1-c^{2}\right)$, in terms of $\kappa$ (gradient energy coefficient) 
 write down the surface energy for an isotropic interface. (2marks)
\end{itemize}

\begin{itemize}
 \item If i were to tilt the free-enery wells of the function $f\left(c\right)=c^{2}\left(1-c^{2}\right)$ by a function
 $L\dfrac{\left(T-T_m\right)}{T_m}c$, will it change the equilibrium of the system? Justify your answer. (2marks)
 \item Assume, an appropriate tilting function, and compute how the region of compositions between where there
 is a immiscibility, changes as a function of temperature. (3marks)
\end{itemize}

\section{Allen-Cahn model}
\begin{itemize}
 \item In the Allen-Cahn model discussed in class, prove that the value of the functional decreases montonically in time, using
 the evolution equation derived for the order parameter. Is it possible to model nucleation using such a deterministic evolution equation? (3marks)
 \item If we were to choose a potential of the form $f\left(\eta\right)=\eta\left(1-\eta\right)$, in terms of $\kappa$, what would
 be the expression that derives you the surface energy. Is there something that needs to be done in "addition" to have stable evolution of
 the interface, given that there is no minima at $\eta=1$ and $\eta=0$. (2marks)
 \item How does the condition of equilibrium at the interface differ from that of the Cahn-Hilliard equation? (1mark)
\end{itemize}

\section{Phase-transformation:pure material solidification}

\begin{itemize}
 \item Draw the steady-state temperature profile across a 1D solid-liquid interface, that is moving with a constant velocity V. 
 What is the assumption of local thermodynamic equilibrium, 
 and what does it imply with respect to the interface temperature. (2marks)
 \item Write down the evolution equation for the temperature field, and mark the source term and the diffusive flux. (1mark)
 \item Expand the free energy of the solid and liquid until second order in temperature, and derive the driving force for
 solidification. Correspondingly derive the modified equation for temperature evolution. (3marks)
\end{itemize}

\section{Phase-transformation: binary alloy solidification}

\begin{itemize}
 \item Draw the profiles of the composition and the diffusion potential across the equilibrium interface derived out of the 
 stationary solutions to the coupled Allen-Cahn and mass-conservation equations. (2marks)
 \item If you convert the mass-conservation equation to an evolution equation for the diffusion potential, what are 
 the diffusive flux and the source terms that result. (1mark)
 \item Point out the analogy between the models for solidification of a binary alloy and a pure material. Relate the 
 source term in the case of a binary alloy to an equilibrium phase-diagram.(1mark)
\end{itemize}

\end{document}
