\documentclass[a4paper,10pt]{article}
\usepackage[utf8]{inputenc}
% \usepackage{amsmath}
\usepackage{amsfonts, amsmath, amsthm, amssymb} % For math fonts, symbols and environments
%opening
\title{Assignment01: Modeling and simulation methods in materials science (MT218)}
\author{}

\begin{document}

\maketitle

\section{Numerical methods}
\begin{itemize}
 \item Using the taylor series expansion work out an fourth order accurate in 
 space operator for $\dfrac{\partial ^{2}}{\partial x^{2}}$.
 \item Utilizing this solve the simple diffusion equation in an \textit{explicit manner}:
 $\dfrac{\partial c}{\partial t} = D\dfrac{\partial^{2}c}{\partial x^{2}}$
 \item How many boundary points would you require in this discretization?
 \item Compare the solutions as functions of time with discretization used 
 in class that was second order accurate in space.
 \item Write down the system of equations that you would require to solve in 
 a \textit{implicit} discretization.
 \item Solve the system of equations using a Jacobi/Gauss-siedel methods
 \item Again compare the solutions against those obtained with the explicit/implicit
 discretizations which are second order accurate.
 \item Can you think of a scheme of discretization which is second order accurate
 in space and time? Will the memory usage of this scheme be higher/lower compared
 to a simple Euler scheme?
\end{itemize}

\section{Cahn-Hilliard equation}
\begin{itemize}
 \item In the example discussed in class, can you work out the evolution for
 the different sinusoidal modes as a function of time. Do all modes grow similarly?
 \item Construct a different symmetric function and work out the equilibrium 
 profiles both numerically and analytically and compare them as was performed in the 
 class? What is the least order of the function that would give you phase separation?
 \item In the equations discussed in class, if the mobility (M) were a function of
 composition, how would you modify the discretization?
 \item Using this impose variable mobilities in the phases(different compositions)
 and see how the evolution of the compositions change with respect to equal 
 mobilities
 \item Add a temperature dependent term ($T-T_M$)*f(c), to the free-energy $c^{2}\left(1-c^{2}\right)$ 
 in order to tilt the wells energetically. Does a linear function $f(c)$ modify the equilibrium points 
 at all? Construct higher order functions which perform this tilting function.
 \item How do you find the equilibrium compositions of the two phases at
 a given temperature?
 \item If you were to initialize the compositions of the phases corresponding to 
 $T-T_M =0$, and then change the temperature to lower and higher values, what 
 happens with respect to the motion of the interface?
 \item Is there a limit, beyond which the system becomes unstable to phase 
 separation? Numerically evaluate this limit. Can you compute it also 
 analytically?
 \item Perform simulations at different temperatures in the stable regime (no phase separation)
 \item Can you device a scheme for measuring the velocities at the interface?
 \item Plot the variation of the velocities for different values of the temperature.
 \item Try an extension of the formulation to two dimensions. Impose perturbations 
 on the system and see whether you get some interesting looking patterns.
 \item If you were to extend the formulation to having three components how many additional 
 equations will you need to solve. Can you hypothetically think of extending the formulation of 
 $c^{2}\left(1-c^{2}\right)$ to $c_A^{2}\left(1-c_A^{2}\right) + c_B^{2}\left(1-c_B^{2}\right) + 
 c_C^{2}\left(1-c_C^{2}\right)$,
 A,B,C being the components. Plot the function and make a decision. 
\end{itemize}



\end{document}
